\raggedright
\chapter{General Information and Training}

\section{Introduction}
The UNL Water Sciences Laboratory (WSL) provides state-of-the-art analytical facilities and equipment for water-related research across the University of Nebraska system. Students, staff, and faculty using the facility may be quite familiar with available analytical services and equipment, but also must understand procedures and protocols to be followed if they intend to work in the Laboratory. This guide is intended to provide an overview of the equipment, safety precautions, general procedures, and methods used as well as expectation for all users of the facility. Routine standard operating procedures (SOPs) are referenced in this guide. Very often, these protocols and procedures are simply a matter of good laboratory practices (GLP) to be followed in any analytical laboratory. This guide was created in order to promote uniformity and to preserve the quality control (QC) procedures for all data produced at this facility. Established in 1990 to enhance and support NU water research by providing analytical equipment and expertise in environmental and isotopic methods, the facility maintains specialized instrumentation for a wide range of contaminants and for stable isotope mass spectrometry. An experienced technical staff maintains and operates the instrumentation and trains others in its use. The unique mix of advanced technology and technical expertise has helped NU faculty lead in the development and application of new methods for water research. 

\section{Laboratory Training}
Most of the following laboratory resources are discussed in detail in specific WSL standard operating procedures (SOPs). All users, including staff and students, should become familiar with the proper use and care of these resources by reading all applicable SOPs. In addition, the laboratory director staff can answer any questions that arise during daily operations. Training of individuals on specific instrumentation depends upon the level of use. Any individual using WSL equipment is responsible for its calibration and general upkeep. It is the user’s responsibility to learn and understand the proper procedures to be followed when using equipment and to notify the laboratory director or staff of any needed repairs or maintenance.\linebreak
In general, all WSL users must have documented training for the following:
\begin{itemize}
	\item Laboratory, chemical, and compressed gas safety 
	\item Laboratory record keeping
	\item Proper use and locations of common lab equipment (balances, micropipettes, reagents)
	\item Proper use of refrigerators and freezers throughout the building
	\item Locations and proper handling and restocking of commonly used supplies and solvents 
	\item Proper handling and replenishment of compressed gas supplies
	\item Proper cleaning of glassware, equipment, and solvent disposal
	\item General laboratory organization and housekeeping
\end{itemize}

The typical sequence for training new WSL users and staff includes the following sequence (all training must be signed off on and finished prior to working in the laboratory):

\begin{enumerate}
	\item 	EHS Core Safety Training, available through instructor or on-line at http://ehs.unl.edu/training/online , and includes:
	\begin{itemize}
		\item Core - Injury and Illness Prevention Plan (IIPP)
		\item Core - Emergency Preparedness Training
		\item Core - Bloodborne Pathogens
		\item Core - Chemical Safety Training (4 units)
		\item Personal Protection Equipment (PPE)	
	\end{itemize}
	\item Additional EHS training modules determined by the Training Needs Assessment for EHS-Related Topics
	\item General WSL standard operating procedures (WSLSOP) and EHS Safe Operating Procedures (EHSSOP) listed below. 
	\item Method specific WSL standard operating procedures (WSLSOP) related to the equipment and methods they will use. 
	\item A scheduled appointment with Autumn Longo, alongo2@unl.edu, to assess laboratory techniques and to complete a written exam.  
\end{enumerate}

